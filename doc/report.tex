\documentclass[a4paper, 10pt]{article}

\usepackage[portuges]{babel}
\usepackage[utf8]{inputenc}
\usepackage[margin=0.9in]{geometry}
\usepackage{amsmath}
\usepackage{graphicx}
\usepackage{datetime}
\usepackage{enumerate}
\usepackage{multicol}
\renewcommand{\baselinestretch}{1.0}

\emergencystretch 1pt%

\title{Relatório EA876 - Trabalho 2}
\author{Larissa Medeiros (178014) e Rafael Gonçalves (186062)}
\date{}
\begin{document}

\maketitle
\begin{multicols*}{2}

\section*{Introdução}

    O objetivo do trabalho foi comparar o tempo de execução de uma tarefa paralelizável - especificamente um filtro de imagens - ao processá-la usando um (linha única de execução) ou mais núcleo (abordagens multi-thread e multi-processo).
    Nós implementamos duas das 


\section*{Método}


\section*{Resultados}


\end{multicols*}
\end{document}


